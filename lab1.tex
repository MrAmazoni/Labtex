\chapter{Сеанс работы в Linux}

\section{Задание}
	Определение понятий и анализ работы ниже названных объектов и процессов, которые происходят при сеансе работы в Linux:
\begin{itemize}
	\item Загрузка системы, ядро ​​системы, регистрация в системе, входящее имя пользователя.
	\item Персональный компьютер, операционная система многих пользователей, пользователи, 		администраторы, домашний каталог.
	\item Учетная запись UID, GID, полное имя, командная оболочка, интерпретатор командной 		строки, задачи администратора и его UID и GID.
	\item Хост, пароль, приглашение командной строки, идентификация.
	\item Виртуальные консоли их интерфейс, вызов виртуальных консолей и работа в них.
	\item Графические консоли.
	\item Имя пользователя, сеанс работы в системе, выходной поток данных, выход из 			системы.
\end{itemize}
Обработка работы команд:
\begin{enumerate}
\item Регистрация в системе
\item Изменение пароля
\item Определение учетной записи пользователя от имени которого выполняется работа
\item Вывод списка пользователей, которые в данный момент зарегистрированы в системе
\item Вывод информации о пользователях, работавших в системе
\item Выход из системы
\end{enumerate}
\section{Основная часть}

\section{Выводы}

